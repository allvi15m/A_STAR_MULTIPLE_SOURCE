\section{Introduction} \label{sec:intro}
With the increase in penetration of Distributed Energy Sources (DERs) in the electric distribution network level, it is becoming increasingly difficult to establish stable network operation based on the traditional centralized control scheme of the grid \cite{WHY}. Especially the integration of photovoltaic and wind generation in the low voltage and medium voltage is making the grid less predictable and difficult to control. In general, today's power grid operation is based on day-ahead planning, in which the balancing of demand and generation of the grid is planned to determine optimum set points of operation. In case of a mismatch with the day ahead planning and real scenario a real-time control is usually put in place to take corrective actions \cite{WHY2}. Due to the increasing uncertainty in the grid these approaches are resulting in the grid operating sub-optimally based on a day-ahead planned control strategy \cite{WHY2}. To tackle this uncertainty of DER integration, integrating energy storage into the grid has been proposed by several researchers \cite{WHY2,ESI1,ES2,ES3}. 
This paper proposes a graph search based control strategy to control multiple energy storage system (ESS). The goal of the control strategy is to provide the most cost optimum operation of the energy storages while providing peak shaving functionality. In \cite{IM1} the researchers propose a control strategy to smooth out the output of a wind farm using the help of ESS. The control makes sure that the output of the wind farm matches a previously predicted output profile. It optimizes the use of the ESS over a prediction horizon to get better results. In \cite{IM2} researchers optimize ESS connected to a distribution feeder hosting a significant amount of renewable energy resources. The objective of the research is to most cost optimally operate the ESS. Reference \cite{IM3} looks at the optimum use of ESS from a transmission grid standpoint. And the researchers in \cite{IM4} show an IEE 14 bus system based proof of concept of an optimum power flow solution considering ESS. From the discussion thus far it is evident that the integration of RES into the distribution grid is creating some new challenges for grid operation and there has been significant research done to show that ESS can be deployed as a possible solution to counteract the uncertainties introduced to the grid by RES integration. To most optimally control the ESS there needs to be a real-time algorithm which takes into account current and predicted system status \cite{gupta_francis_ospina_newaz_2018}. In case of multiple ESS connected to a distribution system, it is also imperative that the solution obtained does not violate any system constraints \cite{rt4}. So an optimum solution considering both system architecture and constraints and current and predicted system status is necessary to most cost optimally control the ESS.